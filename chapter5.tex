\chapter{FUTURE RESEARCH}
\label{chap:future}
There are many areas where the Pingo system can improve.  The first great improvement can happen in the evaluation methodology. All the workflows used to evaluate Pingo were probabilistically generated.   The generation system is designed with flexibility in mind so that it can generate different kinds of workflow loads to the system.  But there is no substitute to real data. Unfortunately, the amount of real data available to the researcher was not enough as to provide good statistical guarantees on the validity of evaluations on it. More real data needs to be gathered in order to produce more accurate evaluations on the performance of the decision algorithms of the system.

Another area of research is the implementation of more sophisticated adaptive decision algorithms that can handle the most diverse types of workloads submitted to the system.  In this respect, this area of research is very much interlinked to previous proposition of gathering a more diverse set of data of workflow submissions. 

Another important area of future research has to do with the design of the system. As we have seen, the system is nothing more than a composition of smaller independent subsystems that poll data from Hadoop or from a database that keeps the state of actions and datasets.  More research is needed on how to tune the parameters that control the frequency of this polling events, so that each independent subsystem carries its own processing computations as effective as possible without putting to much strain in the underlying database cluster.

I am sure that the avid reader of this report will have identified some other opportunities in which the system can be improved or expanded. I gladly accept any related commentaries and suggestions about it. The most rewarding news for me as a researcher is that the system I have created is used and expanded and adapted to different needs by other persons. I certainly have attempted to design it with that goal in mind.