\begin{abstract}
Scientific workflows allow scientists to easily model and express the entire data processing steps, typically as a directed acyclic graph (DAG). These scientific workflows have many tasks that take a long time to compute, and produce a considerable amount of intermediate datasets. Because of the nature of scientific exploration, a scientific workflow is usually modified and re-run multiple times, or new scientific workflows are created that might make use of the intermediate datasets.  Storing intermediate datasets has the potential to save time in computations. Since storage is limited , one main problem that needs a solution is determining which intermediate datasets need to be saved at creation time in order to minimize the computational time of the workflows to be run in the future. In this research thesis I propose the design and implementation of Pingo, a system that is capable of managing the computations of scientific workflows, as well as of predicting what intermediate datasets will be needed by future workflow submissions to the system.
\end{abstract}