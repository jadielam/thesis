\begin{abstract}
In this thesis we propose the design and implementation of a system as the solution to the problem of management of computations and storage of scientific workflow intermediate datasets in fast paced research environments.
Chapter \ref{chap:introduction} introduces the problem, reviews the relevant literature on it, and states what our contributions are towards the solution of the problem.
The interested reader will find Chapter \ref{chap:foundational} the most profitable.  In it I present must of the reasons behind the design of our system.  
Chapter \ref{chap:implementation} might be boring at times, but is equally necessary for a complete exposition of the topic, since it provides a detailed description of the implementation of the system.
Chapter \ref{chap:evaluation} needs to be read with the same care as Chapter \ref{chap:foundational}. In it I propose a methodology to evaluate the performance of the system that might not be obvious at first sight if the reader does not understand well the inner-workings of the system.  If as a reader you find yourself in this position, feel free to go back to Chapter \ref{chap:implementation} to review more throughly the implementation of the system
\end{abstract}