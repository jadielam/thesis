\begin{abstract}
In this thesis we propose the design and implementation of a system as the solution to the problem of management of computations and storage of scientific workflow intermediate datasets in fast paced research environments with constrained storage space.
Chapter \ref{chap:introduction} introduces the problem, reviews the relevant literature on it, and states what our contributions are towards the solution of the problem.
The interested reader will find Chapter \ref{chap:foundational} the most profitable.  In it I present must of the reasons behind the design of our system.
Chapter \ref{chap:implementation} might be boring at times, but is equally necessary for a complete exposition of the topic, since it provides a detailed description of the implementation of the system.  A central section of the chapter is the presentation of two families of decision algorithms that determine which intermediate datasets should be kept in storage at any given time.
Chapter \ref{chap:evaluation} needs to be read with the same care as Chapter \ref{chap:foundational}. In it I propose a methodology to evaluate the performance of the decision algorithms proposed in chapter 4.     We also report on two different experiments that we used to evaluate them, concluding the the adaptive family of algorithms has superior performance in all the analyzed scenarios.
And finally in Chapter \ref{chap:future} we talk about the new possibilities of research that can improve and add to the functionality of the system.
\end{abstract}